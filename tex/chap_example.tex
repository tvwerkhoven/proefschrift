%% Copyright 2010--2014 Tim van Werkhoven (t.i.m.vanwerkhoven@gmail.com)
%% 
%% This work is licensed under the Creative Commons
%% Attribution-Noncommercial-Share Alike 3.0 Unported License. To view a copy
%% of this license, visit http://creativecommons.org/licenses/by-nc-sa/3.0/ or 
%% send a letter to Creative Commons, 171 Second Street, Suite 300, San 
%% Francisco, California, 94105, USA.

% Define image and plotpath for this chapter
\renewcommand{\chapimgpath}{\imgpath chap_example/}
\renewcommand{\chapplotpath}{\imgpath chap_example/}


%%%%%%%%%%%%%%%%%%%%%%%%%%%%%%%%%%%%%%%%%%%%%%%%%%%%%%%%%%%%%%%%%%%%%%%%%%%%%
\chapter{Lorem ipsum dolor eu facilisis nulla}
\label{chap:lorem_ipsum_dolor}
%%%%%%%%%%%%%%%%%%%%%%%%%%%%%%%%%%%%%%%%%%%%%%%%%%%%%%%%%%%%%%%%%%%%%%%%%%%%%

Lorem ipsum dolor sit amet 123 456 789 0, or $123 456 789 0$, consectetur adipiscing elit. Proin vitae volutpat risus. Proin et leo nulla. Donec in risus turpis. Nullam turpis quam, condimentum in varius eu, tempor vel metus \citep{linnik1994c}. Etiam in elit elit, id fermentum nulla. Suspendisse est metus, tempor non laoreet quis, rutrum nec nisl. \marginpar{\emph{Nunc eget felis nec turpis blandit cursus senectus, eros lacus tincidunt est, id pellentesque nisl metus non turpis.}}Curabitur feugiat, dui ac dapibus lobortis, velit turpis feugiat sapien, sed porttitor odio nibh tincidunt lacus. Proin suscipit fermentum ultricies. Sed in \citet{linnik1957c, linnik1994c} odio mauris, eu facilisis nulla. Nunc eget felis nec turpis blandit cursus at vitae urna.

%%%%%%%%%%%%%%%%%%%%%%%%%%%%%%%%%%%%%%%%%%%%%%%%%%%%%%%%%%%%%%%%%%%%%%%%%%%%%
\section[Phasellus eu arcu I]{Phasellus eu arcu I. \hyperlink{toc}{\hyperback}}

\char"E1C0 \char"E1C5 \char"E1CD Nulla iaculis semper mollis\marginpar{``marginpar with options'' at \url{http://tex.stackexchange.com/q/39166}}\marginpar[Pellentesque habitant morbi tristique senectus$\Rightarrow$]{$\Leftarrow$Pellentesque habitant morbi tristique senectus}. Proin in dui in metus suscipit aliquet. Sed congue eleifend augue sit amet ornare. Quisque tincidunt consectetur eros, id vestibulum dui condimentum in. Curabitur vehicula pharetra posuere \citep{rayleigh1902c, kolmogorov1941c, levin1991c}. Donec semper lorem sit amet felis tristique ut imperdiet purus scelerisque. Donec commodo, felis vitae sodales pellentesque, mi nunc malesuada sem \citep{linnik1957c, linnik1994c}, in consectetur sapien enim non risus. 

\begin{quote}
Pellentesque massa orci, 1 to 3 and [4, 6] units viverra vel euismod a, luctus id est. Nam odio dui, viverra in feugiat eget, dapibus ac arcu. Aliquam pulvinar rhoncus nibh, sit amet vulputate mi suscipit sit amet. Vestibulum non neque nulla. Nulla sagittis mollis interdum. Proin aliquet commodo purus. Aenean vehicula erat eget felis molestie semper. Phasellus malesuada rutrum ipsum, in volutpat orci laoreet at. Ut a ante eros. Vivamus luctus imperdiet commodo. Morbi urna risus, sollicitudin id posuere lacinia, feugiat ac velit. Etiam ipsum ante, viverra vitae iaculis at, luctus sed erat. \marginpar[$\Rightarrow$]{$\Leftarrow$}{Duis nec nibh a lacus venenatis pellentesque}Quisque sed nunc magna, et vehicula neque. Etiam mi eros, fermentum a aliquet ut, commodo viverra felis. Donec imperdiet, augue vitae ornare bibendum, sem nulla adipiscing sem, vitae consectetur lectus massa sagittis metus. Mauris suscipit venenatis quam. Quisque consequat semper lacinia. Cras ut erat at ipsum gravida cursus. Duis nec nibh a lacus venenatis pellentesque.
\end{quote}

%%%%%%%%%%%%%%%%%%%%%%%%%%%%%%%%%%%%%%%%%%%%%%%%%%%%%%%%%%%%%%%%%%%%%%%%%%%%%
\section[Nullam interdum massa]{Nullam interdum massa. \hyperlink{toc}{\hyperback}}

Nullam\marginpar[Pellentesque habitant morbi tristique senectus$\Rightarrow$]{$\Leftarrow$Pellentesque habitant morbi tristique senectus} interdum massa quis nisi bibendum tempus. Cras vel ipsum in turpis venenatis placerat vel in \eqnref{eqn:residualrmsptt} leo. Nam sed ipsum et leo tincidunt condimentum. Praesent arcu enim, mattis ac faucibus at, placerat vel tellus. Mauris commodo commodo sodales\marginpar{Vestibulum non neque nulla. Nulla sagittis mollis interdum. Proin aliquet commodo purus. Aenean vehicula erat eget felis molestie semper. Phasellus malesuada rutrum ipsum, in volutpat orci laoreet at.}. Integer venenatis, nibh sed cursus interdum, sem quam posuere risus, sit \eqnref{eqn:residualrmsptt}amet posuere est nisl a justo. Duis interdum dictum libero, at ornare dui mollis nec. Donec bibendum faucibus tempus. Fusce ligula ante, condimentum semper consequat at, blandit eu felis. Curabitur ac venenatis tellus. Duis non risus turpis. 
%%% Equation %%%%%%%%%%%%%%%%%%%%%%%%
\begin{equation}
	\label{eqn:residualrmsptt}
	\sigma_{\varphi, 3}^2 = 0.134 \left ( \frac{d}{\rn} \right )^{5/3}.
\end{equation}
%%% Equation %%%%%%%%%%%%%%%%%%%%%%%%
Proin vitae fringilla risus. In ultrices euismod urna. Suspendisse lacinia ultricies faucibus. Maecenas \eqnref{eqn:residualrmsptt} and $\rn$ power 5/3 and 0.134 pretium ultrices orci fermentum porta. Curabitur varius risus quis ligula semper porttitor. Sed ligula mi, vehicula vel luctus a, ultrices sed tellus. Suspendisse sit amet augue non diam rhoncus volutpat. Aliquam erat volutpat. Etiam eu neque ac nunc imperdiet sodales ac ut risus. Sed id ipsum eleifend dolor posuere facilisis \eqnref{eqn:residualrmsptt2} ut vitae dolor. 
%%% Equation %%%%%%%%%%%%%%%%%%%%%%%%
\begin{equation}
	\label{eqn:residualrmsptt2}
	\sigma_{\varphi, 3}^2 = 0.335 \left ( \frac{D}{\rn} \right )^{5/3} N^{-5/6},
\end{equation}
%%% Equation %%%%%%%%%%%%%%%%%%%%%%%%
mauris condimentum ligula sed felis rhoncus a eleifend lectus posuere. Aenean porta porttitor fringilla. Nulla aliquam felis at nibh venenatis et volutpat neque dictum. Maecenas fermentum sagittis placerat. Duis augue velit, dictum in euismod ac, elementum vitae enim. Nam diam felis, porttitor vel lacinia at, ultricies et tellus. Duis eu nibh sit amet magna rutrum iaculis. Nulla eu justo tellus, \eqnref{eqn:residualrmsptt2} at vehicula felis. Proin sollicitudin feugiat eleifend. Integer sodales, arcu non faucibus interdum, odio tellus suscipit est, ut semper tortor sem eu purus. Aliquam viverra aliquet arcu vitae lacinia. Vestibulum tincidunt semper rutrum. In hac habitasse platea dictumst.

%%%%%%%%%%%%%%%%%%%%%%%%%%%%%%%%%%%%%%%%%%%%%%%%%%%%%%%%%%%%%%%%%%%%%%%%%%%%%
\section[Optical setup]{Optical setup. \hyperlink{toc}{\hyperback}}

Figure 11. shows the optical schematic of coherence-gated wavefront sensing. We use a Coherent \href{http://www.coherent.com/products/?1842/Chameleon-Ultra-Family}{Chameleon Ultra II} Ti:Sa \SI{140}{\femto\second} pulsed, near-infrared laser beam as the source, which is spatially filtered and expanded by a f=\SI{50}{\milli\meter} lens, a \SI{25}{\micro\meter} pinhole, and a f=\SI{300}{\milli\meter} collimating lens (BE). The beam is divided by a 50/50 beamsplitter into a \emph{reference beam} (dashed) and an \emph{object beam} (solid). The object beam reflects off a deformable mirror (DM), which is re-imaged by lenses L1 (f=\SI{300}{\milli\meter}) and L2 (f=\SI{200}{\milli\meter}) onto the back aperture of the Nikon objective with a \SI{16}{\milli\meter} working distance. The  light reflected back from the specimen passes back through the objective where the back aperture is de-magnified  and re-imaged onto the wavefront sensing camera \textsc{ccd}2 (Unibrain \href{http://www.unibrain.com/products/VisionImg/Fire_i_580.htm}{Fire-i 580b}, \num{12}bit,  \num{656}$\times$\num{484} pixels each \num{7.40}$\times$\SI{7.40}{\micro\meter}) by lenses L3 (f=\SI{200}{\milli\meter}) and L4 (f=\SI{300}{\milli\meter}). The reference beam is attenuated by a neutral density filter (ND) to match the intensity of the back-reflected object beam and is then combined with the object beam by the 50/50 beamsplitter BS2. A flip mirror FM can be used to image the focal plane with CCD1 (AVT \href{http://www.alliedvisiontec.com/us/products/cameras/firewire/guppy-pro/f-033bc.html}{Guppy Pro F-033b}, \num{12}bit, \num{656}$\times$\num{492} pixels, each \num{9.9}$\times$\SI{9.9}{\micro\meter}), while blocking the reference beam.

%%%%%%%%%%%%%%%%%%%%%%%%%%%%%%%%%%%%%%%%%%%%%%%%%%%%%%%%%%%%%%%%%%%%%%%%%%%%%
\section[The Sun]{The Sun. \hyperlink{toc}{\hyperback}}

The sun is a main sequence star of spectral type G2.
It is at a mean distance of $A=1.496\times10^{11}$~m or $A=$ \SI{1.496e11}{\meter} from the earth.
The distance varies between $1.471\times10^{11}$~m or \SI{1.471e11}{\meter} and $1.521\times10^{11}$~m or \SI{1.521e11}{\meter}, due to the earth's elliptical orbit.
The close proximity of the sun makes it an ideal subject for the study of stars and stellar atmospheres.
It is the only star of which the surface structure can be resolved.
Current advanced ground-based telescopes regularly image structures of $75$\,--\,$\SI{150}{\kilo\meter}$ or \SIrange{75}{150}{\kilo\metre} or even \SIrange[range-phrase = --,range-units=single]{75}{150}{\kilo\metre} in the solar atmosphere.
The arcsecond is often used as a measure of length for fine structure in the solar atmosphere.
One arcsecond corresponds to $710$ and $735$~km or \SIlist{710;735}{\kilo\metre} at the center of the sun's disc at perihelion and aphelion, respectively.

The sun has a mass of\\
$M_\odot=(1.9889\pm0.0003)\times10^{30}$~kg or better\\
$M_\odot=$ \SI[multi-part-units = brackets]{1.9889\pm\pm0.0003e30}{\kilogram} or otherwise\\
\si{\Msun} = \SI[multi-part-units = brackets]{1.9889\pm\pm0.0003e30}{\kilogram}\\
\si{\Msun} = \SI[separate-uncertainty = false]{1.9889\pm\pm0.0003e30}{\kilogram}

and a luminosity of $L_\odot=(3.844\pm0.010)\times10^{26}$~W\@.
It is made up of mostly hydrogen (\SI{73}{\percent} by mass) and helium (\SI{25}{\percent}).
Unlike the earth, the sun consists only of gas, and therefore has no well-defined surface.
The radius of the visible disk is commonly taken as the solar radius $R_\odot=(6.960\pm0.001)\times10^8~\mathrm{m}$.
Height in the solar atmosphere is often measured relative to unity continuum optical depth at $\lambda=500$~nm.
The luminosity $L_\odot$ and the radius $R_\odot$ yield an effective temperature $T_\mathrm{eff}=5778\pm3$~K\@ or $T_\mathrm{eff}$\,=\,\SI{5778\pm3}{\kelvin}\@.


%%%%%%%%%%%%%%%%%%%%%%%%%%%%%%%%%%%%%%%%%%%%%%%%%%%%%%%%%%%%%%%%%%
\section{Microscopy}
%%%%%%%%%%%%%%%%%%%%%%%%%%%%%%%%%%%%%%%%%%%%%%%%%%%%%%%%%%%%%%%%%%
\begin{figure}[!ht]
	\centering
	\includegraphics[width=0.85\textwidth]{{{\chapimgpath tissue_aberrations-abb_overview.x1a}.pdf}}
	\caption{
	Illustration of sample-induced aberrations in microscopy.
	These aberrations deform the point spread function, degrading both the resolution as well as the returned fluorescence.
	\emph{Left:} in absence of any aberrations, the focus is diffraction limited.
	\emph{Middle:} when focussing inside a sample, the light is aberrated due to heterogeneity in the sample, giving a smeared out focus with loss of signal and resolution.
	\emph{Right:} to mitigate this, one can pre-compensate the incident wavefront on the sample such that the compensation and aberration together again yield a diffraction-limited focus inside the sample.
	The challenge is to determine the shape to optimally compensate aberrations with.
	\label{fig:tissue_aberr}}
\end{figure}

As mentioned before, adaptive optics is also finding applications in other fields, one of these being microscopy.
While it is the opposite of astronomy in terms of scale, the instruments used in both fields often deal with similar visible light, and both are affected by aberrations.
When imaging with a microscope, light travels through a sample, which spatially varies in refractive index as the Earth's atmosphere does.
In \figref{fig:tissue_aberr} this issue is schematically illustrated.
With light with similar properties, similar optics and comparable aberrations, indeed microscopy is a field where adaptive optics can be used for correction.

%%%%%%%%%%%%%%%%%%%%%%%%%%%%%%%%%%%%%%%%%%%%%%%%%%%%%%%%%%%%%%%%%%
\subsection{System aberrations}
%%%%%%%%%%%%%%%%%%%%%%%%%%%%%%%%%%%%%%%%%%%%%%%%%%%%%%%%%%%%%%%%%%

%In astronomy, an important part of telescope design is site-testing, where various prospective sites are characterised in, among other things, seeing.
%In microscopy, this has so far not been done thoroughly.
%
%Since adaptive optics in microscopes often use home-built systems, alignment of such systems is important.
%To 
\begin{figure}[ht]
	\centering
	\includegraphics[scale=0.75]{{{\chapimgpath tissue_aberrations-sph_abb.x1a}.pdf}}
	\caption{
	Because of the difference in refractive index between immersion fluid ($n\tsub{imm}$) and the sample ($n\tsub{sample}$), the converging beam suffers from spherical aberration, where focus position differs between paraxial and peripheral rays, the strength of which increases with depth.
	Such aberrations are not strictly due to the sample, but to a sample-system mismatch, and therefore only vary axially.
	The effect is exaggerated here for clarity.
	\label{fig:sys_abb_depth}}
\end{figure}

There is a distinction between system and sample aberrations, where the former are due to setup misalignments, mirror non-flatness, etc.\ and the latter are due to the sample, as described before.
Ideally, the setup used is well-aligned and delivers diffraction-limited performance.
In practice, this is not always the case, which is why significant gain can already be obtained by correcting the system aberrations with adaptive optics.
After correction of these system aberrations, specimen-induced aberrations can be corrected to improve the resolution further.


For our analysis, we use a linear limb darkening law from \citet{claret2011}:
\begin{equation}
	I(\mu) = 1 - u(1-\mu),
\end{equation}
with $u$ is the limb darkening coefficient, $\mu = \cos(\gamma)$, where $\gamma$ is the angle between the line of sight and the surface normal.
For stars such as \swtgtshort ($T\tsub{eff} \sim \SI{4500}{\kelvin}$, $\log g \sim \num{4.4}$, $[\element{Fe}/\element{H}] \sim \num{0}$), \citet{claret2011} find limb darkening coefficient $u = \num{0.8 \pm 0.05}$ in the \swasp band.
We rewrite $\mu$ as linear coordinate $r = \sqrt{1-\mu^2}$ from \num{0} at the centre of the disk to \num{1} at the limb to yield:
\begin{equation}
	I(r) = 1 - u ( 1 - \sqrt{1-r^2}).
\end{equation}
The flux in a small vertical strip of width $dx$ at the centre of the stellar disk is then given by
\begin{align}
	\nonumber F\tsub{strip} &= dx \int_{-R}^{R} I(r/R) \, dr\\
	                        &= dx \,2 \, R ( 1 - u + \frac{\mathrm\pi u}{4}),
\end{align}
while the total flux is given by 
\begin{align}
	\nonumber F\tsub{total} &= \int_0^R 2 \, \mathrm\pi \, r \, I(r/R) \, dr \\
	                        &= \frac{1}{6} \, R^2 \, \mathrm\pi\,  (6 - 2 u),
\end{align}
with $R$ the stellar radius.

\begin{table}
    \centering
    \caption{
    Inflection points in the \swtgtshort light curve gradient indicative of opacity change.
    Each inflection point indicates a change in gradient, which in turn implies change in opacity, and thus a new ring beginning to transit the star.
    Time is shown in MHJD$-$\num{123}.
    The numbers marked with an asterisk are approximate, indicating the presence of an inflexion point during daylight hours but whose presence is implied by the light curves in adjacent photometric measurements.
    See also \figref{fig:light_curve_overview}.
    \label{tab:lc_inflection}}
    % See https://chenfuture.wordpress.com/2007/09/20/latex-tabular-more/
    % Do not use '\rule[0.35cm]{0cm}{0cm}' to space rows
    %\renewcommand{\tabcolsep}{1cm}
    \renewcommand{\arraystretch}{1.05}
    \newcommand{\marknodata}{{--}}
    \newcommand{\numfunc}[1]{#1}
    \newrobustcmd\markimplied{\tsup{*}}
    \sisetup{
    	table-align-text-post = false,
%    	table-number-alignment = center,
		table-figures-integer = 2,
		table-figures-decimal = 3,
	    table-space-text-post = \markimplied
    }
    \begin{tabular}{SSSSS}
        \toprule
	    \multicolumn{5}{c}{Inflection point [\si{\day}]} \\
	    \midrule
		\multicolumn{3}{c}{{Ingress}} & \multicolumn{2}{c}{{Egress}} \\
		\cmidrule(rl){1-3}
		\cmidrule(rl){4-5}

\numfunc{-52.5}\markimplied 	& \numfunc{-15.15}				& \numfunc{-9.5}\markimplied 	& \numfunc{5.5}\markimplied 	& \numfunc{11.02} \\ 
\numfunc{-51.5}\markimplied 	& \numfunc{-15}					& \numfunc{-8.97}				& \numfunc{8.5}\markimplied 	& \numfunc{11.5}\markimplied  \\ 
\numfunc{-23.5}\markimplied 	& \numfunc{-14.5}\markimplied	& \numfunc{-8.5}\markimplied 	& \numfunc{8.77}				& \numfunc{14.5}\markimplied  \\ 
\numfunc{-22.5}\markimplied 	& \numfunc{-14.15}				& \numfunc{-7.5}\markimplied 	& \numfunc{9.5}\markimplied 	& \numfunc{15.5}\markimplied  \\ 
\numfunc{-21.5}\markimplied 	& \numfunc{-13.5}\markimplied	& \numfunc{-7.08}				& \numfunc{9.83}				& \numfunc{17.5}\markimplied  \\ 
\numfunc{-16.5}\markimplied 	& \numfunc{-12.5}\markimplied	& \numfunc{-6.5}\markimplied 	& \numfunc{9.99}				& \numfunc{24.5}\markimplied  \\ 
\numfunc{-16.155}				& \numfunc{-11.13}				& \numfunc{-6.17}				& \numfunc{10.5}\markimplied 	& \numfunc{31.5}\markimplied  \\ 
\numfunc{-15.5}\markimplied 	& \numfunc{-10.5}\markimplied	& \numfunc{-5.5}\markimplied 	& \numfunc{10.86}				& \numfunc{32.5}\markimplied  \\ 

		\bottomrule
    \end{tabular}
\end{table}

\begin{table}
    \centering
    \caption{
    Measured light curve gradients greater than \SI{0.1}{\per\day} and derived circular orbital speed.
    Time is shown in MHJD$-$\num{123}.
    \label{tab:lcgrad_parms}}
    % See https://chenfuture.wordpress.com/2007/09/20/latex-tabular-more/
    % Do not use '\rule[0.35cm]{0cm}{0cm}' to space rows
    %\renewcommand{\tabcolsep}{1cm}
    \renewcommand{\arraystretch}{1.05}
    \newcommand{\marknodata}{{--}}
    \newcommand{\numfunc}[1]{#1}
    \sisetup{
%    	table-align-text-post = false,
%    	table-number-alignment = center,
%		table-figures-integer = 2,
%		table-figures-decimal = 3,
		separate-uncertainty = false,
    }
    %\sisetup{separate-uncertainty = false}
%for i in np.vstack([sysp['TIME'], sysp['LCGRAD']*3600*24, sysp['LCGRAD_ERR']*3600*24]).T:
%    print "%.4g & \\num{%.2g \pm %.1g} \\\\" % tuple(i)

    \begin{tabular}{@{}SSSSSS@{}}
        \toprule
        {Time} &
        {Gradient} &
        {Speed} &
        {Time} &
        {Gradient} &
        {Speed} \\
        {[\si{\day}]} &
        {[\si{\per\day}]} &
        {[\si{\kilo\meter\per\second}]} &
        {[\si{\day}]} &
        {[\si{\per\day}]} &
        {[\si{\kilo\meter\per\second}]} \\

%		Time & Gradient & Speed \\
%		[\si{\day}] & [\si{\per\day}] & [\si{\kilo\meter\per\second}] \\
        \cmidrule(r){1-3}
        \cmidrule(l){4-6}

\numfunc{-52.00} 	& \numfunc{-0.16 \pm 0.02} 	& \numfunc{-2.0 \pm 0.3} 	& \numfunc{-7.12} 	& \numfunc{-0.86 \pm 0.07} 	& \numfunc{-11 \pm 2} 	\\ 
\numfunc{-23.10} 	& \numfunc{-0.16 \pm 0.02} 	& \numfunc{-2.1 \pm 0.4} 	& \numfunc{-6.96} 	& \numfunc{0.87 \pm 0.02} 	& \numfunc{11 \pm 1} 	\\ 
\numfunc{-22.92} 	& \numfunc{-0.41 \pm 0.03} 	& \numfunc{-5.2 \pm 0.7} 	& \numfunc{-6.23} 	& \numfunc{-0.9 \pm 0.1} 	& \numfunc{-11 \pm 2} 	\\ 
\numfunc{-22.02} 	& \numfunc{0.27 \pm 0.009} 	& \numfunc{3.4 \pm 0.4} 	& \numfunc{-6.04} 	& \numfunc{0.17 \pm 0.01} 	& \numfunc{2.1 \pm 0.3}	\\ 
\numfunc{-19.10} 	& \numfunc{0.16 \pm 0.03} 	& \numfunc{2.0 \pm 0.4} 	& \numfunc{7.97} 	& \numfunc{0.33 \pm 0.04} 	& \numfunc{4.2 \pm 0.7}	\\ 
\numfunc{-16.19} 	& \numfunc{-0.97 \pm 0.06} 	& \numfunc{-12 \pm 2} 		& \numfunc{8.75} 	& \numfunc{0.63 \pm 0.06} 	& \numfunc{8 \pm 1} 	\\ 
\numfunc{-16.09} 	& \numfunc{0.62 \pm 0.07} 	& \numfunc{8 \pm 1} 		& \numfunc{9.75} 	& \numfunc{0.32 \pm 0.05} 	& \numfunc{4.0 \pm 0.8}	\\ 
\numfunc{-15.05} 	& \numfunc{-1.19 \pm 0.04} 	& \numfunc{-15 \pm 2} 		& \numfunc{9.93} 	& \numfunc{-0.87 \pm 0.03} 	& \numfunc{-11 \pm 1}	\\ 
\numfunc{-14.91} 	& \numfunc{1.72 \pm 0.04} 	& \numfunc{22 \pm 3} 		& \numfunc{10.05} 	& \numfunc{-0.26 \pm 0.07} 	& \numfunc{-3 \pm 1} 	\\ 
\numfunc{-14.19} 	& \numfunc{-0.32 \pm 0.04} 	& \numfunc{-4.0 \pm 0.7} 	& \numfunc{10.77} 	& \numfunc{-0.58 \pm 0.03} 	& \numfunc{-7 \pm 1} 	\\ 
\numfunc{-14.04} 	& \numfunc{3.04 \pm 0.06} 	& \numfunc{38 \pm 5} 		& \numfunc{10.94} 	& \numfunc{0.36 \pm 0.02} 	& \numfunc{4.6 \pm 0.6}	\\ 
\numfunc{-11.19} 	& \numfunc{0.52 \pm 0.05} 	& \numfunc{7 \pm 1} 		& \numfunc{11.06} 	& \numfunc{1.09 \pm 0.09} 	& \numfunc{14 \pm 2} 	\\ 
\numfunc{-10.97} 	& \numfunc{-0.52 \pm 0.01} 	& \numfunc{-6.6 \pm 0.8} 	& \numfunc{14.90} 	& \numfunc{-0.15 \pm 0.01} 	& \numfunc{-1.9 \pm 0.3}\\ 
\numfunc{-10.18} 	& \numfunc{-0.54 \pm 0.03} 	& \numfunc{-6.8 \pm 0.9} 	& \numfunc{16.82} 	& \numfunc{-0.19 \pm 0.02} 	& \numfunc{-2.4 \pm 0.4}\\ 
\numfunc{-9.98} 	& \numfunc{-0.27 \pm 0.01} 	& \numfunc{-3.5 \pm 0.5} 	& \numfunc{23.77} 	& \numfunc{0.88 \pm 0.05} 	& \numfunc{11 \pm 2} 	\\ 
\numfunc{-9.04} 	& \numfunc{-0.57 \pm 0.03} 	& \numfunc{-7 \pm 1} 		& \numfunc{30.85} 	& \numfunc{-0.28 \pm 0.03} 	& \numfunc{-3.5 \pm 0.6}\\ 
\numfunc{-8.90} 	& \numfunc{0.66 \pm 0.06} 	& \numfunc{8 \pm 1} 		& \numfunc{31.88} 	& \numfunc{-0.14 \pm 0.03} 	& \numfunc{-1.7 \pm 0.4}\\ 
\numfunc{-8.12} 	& \numfunc{-0.87 \pm 0.02} 	& \numfunc{-11 \pm 1} 		& \numfunc{42.80} 	& \numfunc{-0.22 \pm 0.02} 	& \numfunc{-2.8 \pm 0.4}\\


		\bottomrule
    \end{tabular}
\end{table}

%%%%%%%%%%%%%%%%%%%%%%%%%%%%%%%%%%%%%%%%%%%%%%%%%%%%%%%%%%%%%%%%%%%%%%%%%%%%%
\section[Testing]{Testing. \hyperlink{toc}{\hyperback}}

\copyright~Alfred G. de Wijn\\
The radius of the visible disk is commonly taken as the solar radius $R_\odot=(6.960\pm0.001)\times10^8$~m.\\
The 1998 observations targeted plage around NOAA active region 8235 at a heliocentric viewing angle of $40$~degrees (µ = 0.77) in the $1600$~\AA\ UV passband and in the EUV $171$~\AA\ (\ion{Fe}{IXX}) and $195$~\AA\ (\ion{Fe}{XII}) passbands.

Regular text: \ion{C}{IV}\ \ion{H$\alpha$}{}\\
\textbf{Boldface text: \ion{C}{IV}\ \ion{H$\alpha$}{}}\\
\emph{Emphasized text: \ion{C}{IV}\ \ion{H$\alpha$}{}}\\
\url{http://url.foo.bar/~bla}\\
\texttt{Typewriter text.}

Some math characters: $(x,y)$, $i=1,2,3$, $\Delta\phi\approx45$~degrees also $\approx$\,\SI{45}{\degree}, $T\sim10^4$~K, $k_h$, $x^2$, arcsec$^{-1}$.
\begin{equation}
	p_1(\tau)=\left\{\begin{array}{ll}
		\displaystyle0&\kern1em \mathrm{if}~\tau\le t_\mathrm{obs}\,,\\
		\displaystyle\frac{\tau-t_\mathrm{obs}}{\tau+t_\mathrm{obs}}&
			\kern1em \mathrm{if}~\tau\ge t_\mathrm{obs}\,.
		\end{array}\right.
\end{equation}
\begin{equation}
	N'_i=
		N\int_0^\infty\!\!p_i(\tau)\,
		D_\lambda(\tau)\,\mathrm{d}\tau\,,
\end{equation}

Maecenas a eros odio. Nulla odio arcu, ultrices eu pellentesque in, auctor eu est. Sed accumsan tortor eget enim fringilla aliquam. Aliquam tempus neque at mauris interdum ut eleifend sem viverra. Donec commodo congue lobortis. Suspendisse potenti. Morbi nec augue quis lorem dignissim iaculis vel vitae augue. Aenean eu libero nunc. Cum sociis natoque penatibus et magnis dis parturient montes, nascetur ridiculus mus. Ut in tellus eget quam gravida vestibulum. Pellentesque habitant morbi tristique senectus et netus et malesuada fames ac turpis egestas. Nulla porta eros vitae augue lobortis et varius nulla ornare. Mauris a odio ipsum, sit amet volutpat nisl. Vivamus vulputate, ligula dignissim gravida pretium, metus sapien ultrices diam, ac commodo justo magna eget massa. Aliquam ac est et massa tristique accumsan. 

\begin{Verbatim}[numbers=left, framerule=1pt]
% see http://www.ctan.org/tex-archive/macros/xetex/latex/xltxtra/xltxtra.pdf
\usepackage{xltxtra} 
%\usepackage{fontspec} % loaded by xltxtra
\setmainfont{Cambria}
\setmonofont{Monaco}
\setsansfont{Minion Pro}
%\usepackage{xunicode} % loaded by xltxtra
%\usepackage{metalogo} % loaded by xltxtra
%\usepackage{fixltx2e} % loaded by xltxtra
\end{Verbatim}

%%%%%%%%%%%%%%%%%%%%%%%%%%%%%%%%%%%%%%%%%%%%%%%%%%%%%%%%%%%%%%%%%%%%%%%%%%%%%
\section{Donec cursus}

Donec cursus suscipit elit vestibulum porta. Maecenas cursus sapien a urna ornare vel lacinia risus tristique. Nunc eu dolor sit amet felis sagittis sagittis eu vel sem. Suspendisse eleifend, neque malesuada tristique ultrices, mi ligula ornare velit, non rhoncus libero quam suscipit diam. Maecenas malesuada dui dolor. In hac habitasse platea dictumst. Nam id enim lectus, in bibendum metus. Suspendisse imperdiet lectus imperdiet velit feugiat aliquet. Nulla rhoncus libero nec libero cursus tincidunt. Donec aliquam pulvinar posuere. Sed enim mauris, varius et convallis accumsan, imperdiet ut nulla. Aenean orci felis, dapibus in tempus nec, fringilla in nisi. 

\providecommand{\mcol}[2]{\multicolumn{#1}{c}{#2}}
\providecommand{\mcoll}[2]{\multicolumn{#1}{@{}l@{}}{#2}}

\providecommand{\emptycell}{\mcol{3}{---}}
% This is a hyphen minus for negative numbers
% Hyphen minus: -, minus: −, dash: -
\providecommand{\minussign}{-}

\providecommand{\tabval}[3]{#1.&#2 & ± #3}
\providecommand{\tabvalb}[4]{#1.&#2&\makebox[0pt][l]{\tsub{$-$#3}}\tsup{+#4}}

\begin{table}
	% See https://chenfuture.wordpress.com/2007/09/20/latex-tabular-more/
	% Do not use '\rule[0.35cm]{0cm}{0cm}' to space rows
	%\renewcommand{\tabcolsep}{1cm}
	\renewcommand{\arraystretch}{1.2}
	
	\begin{centering}
		% col 1 (l): parameter
		\caption{
		Example of semi-complex table with row-spanning and column-spanning.
		\textbf{N.B. This is better solved using the SI packge!}
		}
		% col 2 (l): parameter unit
		% col 3 - 5 (r@{}l@{}l): 2 NEI - RGS, value &.decimal & ± error
		% col 6 - 8 (r@{}l@{}l): 2 NEI - MOS,  -- " -- 
		% col 9 - 11 (r@{}l@{}l): 2 NEI/cool - RGS,  -- " -- 
		% col 12 - 14 (r@{}l@{}l): 2 NEI/cool - MOS,  -- " -- 
		
		\begin{tabular}{llr@{}l@{ }lr@{}l@{ }lr@{}l@{ }lr@{}l@{ }l}
		
		 & & \mcol{6}{2 \textsc{NEI}} & \mcol{6}{2 \textsc{NEI}, 1 cooling} \\ 
		 \cmidrule(l){3-8}
		 \cmidrule(l){9-14}

		\mcol{2}{Parameter} & \mcol{3}{RGS} & \mcol{3}{MOS} & \mcol{3}{RGS} & \mcol{3}{MOS}  \\ 
		\midrule
		
		N\tsub{h} & (10\tsup{21}\,cm\tsup{\minussign 2}) %
			& \mcol{3}{1.14} & \mcol{3}{1.14} & \mcol{3}{1.14} & \mcol{3}{1.14} \\ 
		n\tsub{e}n\tsub{h}V\tsub{1} & (10\tsup{58} cm\tsup{\minussign 3}) %
			& \tabval{20}{5}{0.5} & \tabval{14}{9}{0.2} & \emptycell & \tabval{11}{4}{0.1}\\
		n\tsub{e}n\tsub{h}V\tsub{2} & (10\tsup{58} cm\tsup{\minussign 3}) %
			& \tabval{99}{6}{3.7} & \tabval{103}{5}{1.6} & \emptycell & \tabval{138}{5}{1.6}\\
		kT\tsub1 & (keV) %
			& 0.&\mcoll{2}{85 (fixed)}& \tabval{0}{85}{0.01} & 0.&\mcoll{2}{85 (fixed)} & \tabval{0}{85}{0.01}\\
		n\tsub{e}t\tsub{1} & (10\tsup{10} cm\tsup{\minussign 3} s) %
			& \tabval{4}{67}0.32{} & \tabvalb{7}{20}{0.17}{0.06} & \tabval{2}{32}{0.13} & \tabval{6}{7}{0.2}\\	 
		kT\tsub{2} & (keV) %
			& \tabval{0}{12}{0.00} & \tabval{0}{22}{0.00} & \tabval{0}{15}{0.00} & \tabval{0}{18}{0.00}\\
		n\tsub{e}t\tsub{2} & (10\tsup{10} cm\tsup{\minussign 3} s) %
			& \tabval{41}{2}{6.6} & \tabvalb{52}{1}{3.9}{610} & \tabval{54}{5}{3.7} & \tabval{99}{6}{2.7}\\
		\\
		\mcol{2}{Element} & \mcol{12}{Abundance (wrt solar)}  \\
		\midrule
		C &  & \tabval{0}{55}{0.08} & \emptycell               & \tabval{0}{18}{0.03} & \emptycell \\
		N &  & \tabval{0}{06}{0.02} & \emptycell               & \tabval{0}{07}{0.03} & \emptycell \\
		O &  & \tabval{0}{20}{0.01} & \tabval{0}{19}{0.01} & \tabval{0}{26}{0.03} & \tabval{0}{23}{0.01} \\
		Ne & & \tabval{0}{22}{0.02} & \tabval{0}{22}{0.02} & \tabval{0}{33}{0.03} & \tabval{0}{26}{0.01} \\
		Mg & & \emptycell               & \tabval{0}{31}{0.02} & \emptycell               & \tabval{0}{30}{0.02} \\
		Si & & \emptycell               & \tabval{0}{25}{0.03} & \emptycell               & \tabval{0}{21}{0.03} \\
		S  &  & \emptycell               & \tabval{0}{33}{0.09} & \emptycell               & \tabval{0}{34}{0.09} \\
		Fe & & \tabval{0}{26}{0.01} & \tabval{0}{29}{0.01} & \tabval{0}{38}{0.02} & \tabval{0}{42}{0.01} \\

%\tabval{}{}{}
	\end{tabular}
   \label{tab:abundances}
\end{centering}
\end{table}

\begin{center}
\captionof{table}[short caption]{%
\kant[1-2]}
\end{center}

Suspendisse congue tortor sit amet eros volutpat pellentesque. Duis semper nisl nunc, tristique semper tellus. Cras rhoncus erat vitae est dignissim sit amet luctus tortor pellentesque. Suspendisse potenti. Nullam eleifend hendrerit lorem sed tristique. Vivamus a venenatis leo. Donec sem nunc, dapibus vel lobortis vel, interdum vitae quam. Cum sociis natoque penatibus et magnis dis parturient montes, nascetur ridiculus mus. Sed ut purus quis dolor ullamcorper laoreet ut quis magna. Pellentesque habitant morbi tristique senectus et netus et malesuada fames ac turpis egestas. Nunc commodo laoreet justo facilisis sodales. 

%%%%%%%%%%%%%%%%%%%%%%%%%%%%%%%%%%%%%%%%%%%%%%%%%%%%%%%%%%%%%%%%%%%%%%%%%%%%%
\section{Phasellus eu arcu}

Phasellus eu arcu lacus. Sed vel convallis sem. Donec vel nibh eget dolor scelerisque volutpat a quis arcu. Nunc ut velit nisl. Vestibulum in arcu vitae magna mattis molestie vitae eu ante. Nam eu dui felis, sed vulputate sapien. Phasellus eu sollicitudin orci. Nunc lobortis ipsum in dui laoreet ac euismod purus lacinia. Curabitur in eros erat, eu mattis nisl. Donec at mi nisl. Aenean arcu velit, vulputate ut tempor non, commodo gravida nisl. Donec hendrerit volutpat mollis. Vestibulum tristique, arcu ac pulvinar feugiat, odio nibh facilisis risus, ac fermentum purus ante sit amet turpis. Morbi id aliquam sapien.

%%%%%%%%%%%%%%%%%%%%%%%%%%%%%%%%%%%%%%%%%%%%%%%%%%%%%%%%%%%%%%%%%%%%%%%%%%%%%
\subsection{Phasellus vehicula}

Phasellus vehicula cursus felis. Sed a leo nibh, nec congue justo. Cras nisi arcu, tristique non accumsan at, tempor non diam. Morbi faucibus ligula interdum neque ultricies adipiscing. Fusce pharetra aliquam magna, sed ultricies purus rhoncus sit amet. Etiam et nisl iaculis tellus congue luctus elementum nec mi. In hac habitasse platea dictumst. Nunc ornare lectus ut neque accumsan ullamcorper. Fusce id lacus sed mi hendrerit convallis. Duis ut ultricies lectus. 

%%%%%%%%%%%%%%%%%%%%%%%%%%%%%%%%%%%%%%%%%%%%%%%%%%%%%%%%%%%%%%%%%%%%%%%%%%%%%
\subsection{Nullam interdum massa}

Nullam interdum massa quis nisi bibendum tempus. Cras vel ipsum in turpis venenatis placerat vel in leo. Nam sed ipsum et leo tincidunt condimentum. Praesent arcu enim, mattis ac faucibus at, placerat vel tellus. Mauris commodo commodo sodales. Integer venenatis, nibh sed cursus interdum, sem quam posuere risus, sit amet posuere est nisl a justo. Duis interdum dictum libero, at ornare dui mollis nec. Donec bibendum faucibus tempus. Fusce ligula ante, condimentum semper consequat at, blandit eu felis. Curabitur ac venenatis tellus. Duis non risus turpis. 
%%% Equation %%%%%%%%%%%%%%%%%%%%%%%%
\begin{equation}
	\label{eqn:3}
	\sigma_{\varphi, 3}^2 = 0.134 \left ( \frac{d}{\rn} \right )^{5/3}.
\end{equation}
%%% Equation %%%%%%%%%%%%%%%%%%%%%%%%
Proin vitae fringilla risus. In ultrices euismod urna. Suspendisse lacinia ultricies faucibus. Maecenas pretium ultrices orci fermentum porta. Curabitur varius risus quis ligula semper porttitor. Sed ligula mi, vehicula vel luctus a, ultrices sed tellus. Suspendisse sit amet augue non diam rhoncus volutpat. Aliquam erat volutpat. Etiam eu neque ac nunc imperdiet sodales ac ut risus. Sed id ipsum eleifend dolor posuere facilisis \eqnref{eqn:residualrmsptt2} ut vitae dolor. 
%%% Equation %%%%%%%%%%%%%%%%%%%%%%%%
\begin{equation}
	\label{eqn:4}
	\sigma_{\varphi, 3}^2 = 0.335 \left ( \frac{D}{\rn} \right )^{5/3} N^{-5/6},
\end{equation}
%%% Equation %%%%%%%%%%%%%%%%%%%%%%%%
such that we obtain 42.
